\chapter{Conclusioni}
\label{Cha:conclusioni}
\thispagestyle{empty}

L'ultimo capitolo fornisce una sintesi dei risultati ottenuti e alcuni suggerimenti su possibili problematiche che potrebbero essere risolto in futuro.

\section{Risultati ottenuti}





%Per verificare il metodo di misurazione, sono stati condotti esperimenti su due pneumatici.
%In questa ricerca viene sviluppato un programma LabVIEW basato sul suo modulo di visione.
%Gli algoritmi sviluppati sono implementati nel software.
%Quando il pneumatico rotola sulla piastra di supporto, vengono trovati il ​​numero e la posizione della scanalatura del pneumatico e la profondità viene calcolata dal programma.
%I risultati sperimentali hanno rilevato che il sistema è in grado di acquisire il profilo della striscia laser della sezione del battistrada del pneumatico e l'algoritmo può identificare la scanalatura del battistrada e misurarne la profondità.
%L'immagine del misuratore di profondità è mostrata in Figura 20, e il metodo di misurazione che utilizza il misuratore di profondità è mostrato nella Figura 21.
%La precisione del pneumatico che utilizza il misuratore di profondità è di 0,01 mm e può soddisfare l'accuratezza della misurazione.
%Il metodo di misurazione basato sulla visione artificiale è mostrato nella Figura 22e l'interfaccia di misurazione basata sulla visione artificiale è mostrata nella Figura 23.
%I risultati della misurazione della visione artificiale ei risultati del misuratore di profondità sono riportati nella Tabella 3 .
%I risultati misurati del profondimetro sono il valore medio dei tre tempi di misura.
%I risultati sperimentali hanno indicato che il sistema di misurazione poteva identificare la scanalatura del battistrada e misurarne automaticamente la profondità.
%L'errore percentuale massimo è 2,47% e l'errore percentuale minimo è 0,41%.
%Tutti gli errori sono inferiori a 0,2 mm e il test potrebbe soddisfare la richiesta di misurazione della profondità del battistrada.


\section{Conclusioni}


\section{Future work}
Future research directions include:


\nocite{opencv_library}
\nocite{helix_toolkit_documentation}
\nocite{wpf_whatis}
\nocite{wrapping_whatis}
\nocite{dll_whatis}
\nocite{direct_industry}
\nocite{redazione_geomedia}
\nocite{lmi_technologies}
\nocite{image_s_blog}
\nocite{point_cloud_library_1}
\nocite{point_cloud_library_2}
\nocite{point_cloud_library_3}
\nocite{point_cloud_library_4}
\nocite{the_gsl_team}
\nocite{wang2019}
\nocite{booksdaglib0034531}
\nocite{gonzalez2008digital}