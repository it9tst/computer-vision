\chapter{Introduzione}
\label{Cha:introduzione}
\thispagestyle{empty}

Viviamo in un mondo 3D con processi di produzione che richiedono un'ispezione delle parti ad alta velocità e altamente accurata. Le soluzioni di visione artificiale progettate per supportare la produzione automatizzata possono essere complicate da progettare, configurare e installare, e spesso richiedono personale specializzato con un background ingegneristico.\\
\newline
Le soluzioni 2D progettate per risolvere i problemi di misurazione 3D sono complicate, costose e spesso non hanno successo. Il 2D semplicemente non può compensare la variazione di altezza senza una sorta di compromesso in termini di costi, qualità o affidabilità.\\
\newline
Sebbene le soluzioni 3D siano tradizionalmente considerate costose o complicate da configurare, la rivoluzionaria famiglia di prodotti Gocator di LMI semplifica la misurazione 3D per l'automazione in fabbrica.\\
\newline
LMI Technologies Inc. (LMI) è leader mondiale nella tecnologia dei sensori smart 3D per applicazioni di ispezione e di misura in ambito industriale ma non solo. LMI sviluppa costantemente nuove soluzioni 3D che consentano agli utenti finali di sfruttare gli enormi vantaggi della tecnologia laser applicata al controllo di qualità, all'ottimizzazione dei materiali per ridurre gli sprechi oppure all'automazione nelle moderne realtà industriali, al fine di migliorarne la produttività.\\
\newline
Il progetto ha l'obiettivo di interfacciarsi con il Gocator e di creare una demo per \textit{PC}, comprensiva di \textit{GUI}, che permetta l'elaborazione della \textit{point cloud}, effettuando misurazioni sui prodotti presi in esame.\\

\noindent La relazione è articolata come segue:
\begin{itemize}
	\item nel \textbf{capitolo \ref{Cha:gocator}} viene presentato il gocator e il settaggio utilizzato;
	\item nel \textbf{capitolo \ref{Cha:analisi}} vengono descritti i prodotti presi in analisi e gli algoritmi utilizzati;
	\item nel \textbf{capitolo \ref{Cha:desktop}} viene presentato lo sviluppo dell'applicazione desktop e le sue funzionalità;
	\item il \textbf{capitolo \ref{Cha:conclusioni}} conclude la relazione presentando i risultati ottenuti, gli obiettivi raggiunti ed eventuali problematiche che potrebbero essere risolte in futuro;
\end{itemize}
Il codice e l'applicazione di tutto il progetto possono essere reperiti nella seguente repository git: \href{https://github.com/it9tst/computer-vision}{https://github.com/it9tst/computer-vision}